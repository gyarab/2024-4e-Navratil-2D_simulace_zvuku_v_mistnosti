\subsection{Room1Controller}
Třída Room1Controller je velmi podobná třídě Room0controller. Metody: \textit{setStage(Stage stage), initialize(), createTimeline(), getStroke(), getRoomWalls(), getRoomCorners(), getXMin(), getXMax(), getYMin(), getYMax(), initializeLines(double xMin, double xMax, double yMin, double yMax), updateLayout(), createOverlay(), overlayRectangles(), updateTimerLabel(), handleButtonHlavniMenuClick(), handleButtonStopClick(), handleButtonResumeClick(), handleButtonResetClick(), handlePaneClick(MouseEvent event)} plní stejnou funkci jako ve třídě Room0Controller viz. kapitola 4.6. Pouze třída \textit{initializeRectangle(double x, double y)} je rozdílná a to v tom, že místo čtverce se vytvoří obdelník.