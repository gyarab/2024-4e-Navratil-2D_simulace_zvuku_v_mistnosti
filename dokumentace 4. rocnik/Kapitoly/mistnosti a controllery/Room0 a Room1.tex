\subsection{Room0 a Room1}
Tyto dvě třídy jsou zodpovědny za inicializaci počáteční velikosti scény, načítání příslušného \textit{FXML} souboru, vytváření a zobrazování  scény a \textit{stage}, vytváření controlleru a připisování controlleru k \textit{fxml} souboru. Třídy mají tři metody \textit{setScene(Stage stage)}, která se stará o vše výše zmíněné, \textit{ getMistnost0ScreenHeight()}, \textit{ getMistnost0ScreenWidth()}. \textit{Room1} vytváří scénu ze 4. obrázku a \textit{Room1} vytváří scénu velmi podobnou s jediným rozdílem a to že místo čtvercové místnosti je tam obdelníková. 
